%input mp-tool;

%color si_yellow, si_blue, si_red, si_white, si_black;

%si-blue   = cmyk(1,0.6,0,0.1);
%si-red    = cmyk(0,1  ,1,0  );
%si-yellow = cmyk(0,0.1,1,0  );
%si-white  = cmyk(0,0  ,0,0  );
%si-black  = cmyk(0,0  ,0,1  );

%si_blue   = blue;
%si_red    = red;
%si_yellow = yellow;
%si_white  = white;
%si_black  = black;

\setupcolors[state=start]
\definecolor[si-blue]   [c=1,m=.6,k=.1]
\definecolor[si-red]    [m=1,y=1]
\definecolor[si-yellow] [m=.1,y=1]
\definecolor[si-white]  [white]
\definecolor[si-black]  [black]

\setupinteraction
	[author={Mojca Miklavec},
	 title={Slovenski grb}]

\startMPinclusions

%linejoin := mitered;
%linejoin := beveled;
linecap := butt;
numeric unit; unit = 1cm;
numeric linewidth; linewidth = .4pt;
%numeric borderwidth;

% nariše takšno mrežo kot je v originalni geometrijski definiciji grba
def draw_basic_grid =
	drawoptions(withpen pencircle scaled linewidth);
	% okvir
	draw unitsquare shifted (-.5,0) xyscaled (8unit,9unit);
	% navpične črte spodaj
	for i = -3 upto 3:
		draw ((i,0)--(i,3)) scaled unit;
	endfor;
	% navpične črte zgoraj
	for i = -3 step 1.5 until 3:
		draw ((i,3)--(i,9)) scaled unit;
	endfor;
	% vodoravne črte v sredini
	for i = 3 upto 6:
		draw ((-3,i)--(3,i)) scaled unit;
	endfor;
	
	drawoptions();
enddef;

%------------------%
% Zgornji rob grba %
%------------------%
% krožni lok s središčen v (0,0), ki gre skozi točko (2,9)
path arc_upper;
arc_upper =
	fullcircle scaled 2(2++9)  % radij je enak dolžini vektorja (2,9)
	cutbefore ((4,0)--(4,9))   % odreži krog pred desnim robom
	cutafter ((-4,0)--(-4,9)); % odreži krog za levim robom


% levi konec zgornjega krožnega loka
pair arc_upper.l; arc_upper.l = llcorner arc_upper;
% desni konec zgornjega krožnega loka
pair arc_upper.r; arc_upper.r = lrcorner arc_upper;
% pomožna črta, ki nakazuje velikost zgornjega krožnega loka
path arc_upper.helpline; arc_upper.helpline = ((0,0)--(2,9));

%---------------------%
% Spodnja robova grba %
%---------------------%
%
% levi in desni spodnji del krožnega loka:
% - 0: notranji (morje)
% - 1: srednji
% - 2: zunanji
path arc_lower.l[]; pair arc_lower.l.c;
path arc_lower.r[]; pair arc_lower.r.c;
% Na spodnji strani imamo tri krožne loke. Zunanji predstavlja rob grba,
% srednji meji na rdečo in modro, notranji pa na modro in belo.
%
% Krožni loki imajo središče v (-1.5,4) in (1.5,4).
% Notranji krožni lok je tangenten na zgornjo levo (-2,3; 1)
% oz. zgornjo desno (2,3; 1) krožnico, ki oblikuje valove,
% zato je radij tega loka 1 + sqrt(3.5^2+1^2).

arc_lower.l.c = ( 1.5,4);
arc_lower.r.c = (-1.5,4);

% Pomembna količina, ki določa razdalje med valovi in razdalje v robu grba,
% je določena takole:
% - na premico y = 3 skiciramo tri enotske krožnice s središči v
%   (-2,3), (0,3) in (2,3)
% - tik pod njimi skiciramo še dve kročnici, ki se jih tesno dotikata
% - skozi štiri dotikališča med spodnjimi in zgornjimi krožnicami
%   potegnemo novo premico (torej y = 3 - sqrt(3)/2)
% - tretjina razdalje med obema premicama je ena enota, recimo ji 'd',
%   d = 1 / 2*sqrt(3)
%
% 'd' je zdaj razdalja med posameznimi spodnjimi krožnimi loki
% ter razdalja med valovi.

numeric d; d = .5/sqrt(3);

for i = 0 upto 2:
	arc_lower.l[i] =
		fullcircle rotated 180 % only a temporary cut to the left
		scaled 2(d*i+1+(3.5++1)) shifted arc_lower.l.c
		cutafter ((0,0)--(0,-1));
%	arc_lower.r[i] =
%		fullcircle
%		scaled 2(d*i+1+(3.5++1)) shifted arc_lower.r.c
%		cutbefore ((0,0)--(0,-1));
endfor;

% Zunanji in srednji krožni lok na vrhu tangentno prehajata v ravno črto.
pair border.l[].t, border.l[].b;
pair border.r[].t, border.r[].b;
path border.l[]  , border.r[]  ;

% vrh levega zunanjega roba
border.l[2].t = arc_upper.l;
% dno levega zunanjega roba
border.l[2].b =
	% premer kroga je enak razdalji med levim koncem gornjega loka
	% in srediscem spodnjega levega loka
	fullcircle scaled (arclength (arc_upper.l--arc_lower.l.c))
	% njegovo sredisce je ravno med tema dvema tockama
	shifted .5[arc_upper.l,arc_lower.l.c]
	% presecisce med tem krogom in levim spodnjim lokom je tocka
	intersectionpoint arc_lower.l[2];
% levi zunanji rob
border.l[2] = border.l[2].t--border.l[2].b;
% popravi zunanji levi spodnji lok (odrezi presezek)
arc_lower.l[2] := arc_lower.l[2] cutbefore border.l[2].b;
% izracunaj dno levega notranjega roba
border.l[1].b = arc_lower.l[1] intersectionpoint (arc_lower.l.c--border.l[2].b);
% popravi srednji levi spodnji lok (odrezi presezek)
arc_lower.l[1] := arc_lower.l[1] cutbefore border.l[1].b;

border.l[1].t =
	% vzporednica levemu zunanjemu robu skozi spodnjo tocko
	(origin--(border.l[2].t-border.l[2].b)) scaled 2 shifted border.l[1].b
	% presecisce z zgornjim robom
	intersectionpoint arc_upper;
% levi notranji rob
border.l[1] = border.l[1].t--border.l[1].b;

%================%
% DESNA POLOVICA %
%================%
% vse na desni je samo kopija tistega na levi desni robovi so samo kopija levih

% desna ravna robova
for i=1 upto 2:
	border.r[i]   = reverse border.l[i]   reflectedabout ((0,0),(0,1));
	border.r[i].t =         border.l[i].t reflectedabout ((0,0),(0,1));
	border.r[i].b =         border.l[i].b reflectedabout ((0,0),(0,1));
endfor;

% desni spodnji krozni loki
for i=0 upto 2:
	arc_lower.r[i] = reverse arc_lower.l[i] reflectedabout ((0,0),(0,1));
endfor;

% 2 - zunanji (rdeci) obris grba
% 1 - notranji (modri) obris grba
path border[];
border[0] = arc_lower.l[0] -- arc_lower.r[0];

border[1] = arc_upper cutbefore border.r[1].t cutafter border.l[1].t --
	arc_lower.l[1] -- arc_lower.r[1] -- cycle;

border[2] = arc_upper cutbefore border.l[1].t -- arc_lower.l[2] --
	arc_lower.r[2] -- arc_upper cutafter border.r[1].t --
	(reverse border[1] cutafter border.l[1].t) -- cycle;

%tertiarydef a intersection_join b =
% begingroup save t_;
%  (t_1,t_2)=a intersectiontimes b; a.sub(0,t_1)&&b.sub(t_2,infinity)
% endgroup
%enddef;

% Triglav

% kroznice, ki oblikujejo valove
% 1 0 4
%  2 3
path circle_waves[]; pair circle_waves.c[];
% gornji sredinski krog
circle_waves[0] = fullcircle scaled 2 shifted (0,3);
for i = 1 upto 4:
	circle_waves.c[i] = dir(120+60i) scaled 2 shifted (0,3);
	circle_waves[i] = fullcircle scaled 2 shifted circle_waves.c[i];
endfor;

% del enotske kroznice med kotoma a in b
vardef unitcircle_arc(expr a, b) =
%	(fullcircle scaled 2 cutbefore (origin--dir(a)) cutafter (origin--dir(b)))
	((fullcircle scaled 2 cutafter (origin--dir(b-a))) rotated a)
enddef;

path waves.model[], waves.top, waves.mid, waves.bot;
% gornji val
waves.model[0] =
	unitcircle_arc(180,-60) shifted (-2,0) ..
	reverse (unitcircle_arc(60,120) shifted 2dir(-120)) ..
	unitcircle_arc(-120,-90);
waves.model[0] := (waves.model[0] .. reverse waves.model[0] reflectedabout ((0,0),(0,1))) shifted (0,3);
% srednji val (podaljšani gornji val) v položaju zgornjega vala
% ena enota, ki se bo kopirala dalje
path waves.model.template;
waves.model.template = ( reverse (
	unitcircle_arc(90,120) shifted 2dir(-60) ..
	(reverse (unitcircle_arc(-120,-60))) ..
	(unitcircle_arc(60,90) shifted 2dir(-120))
));
	
waves.model.template :=
	(waves.model.template) shifted (-2,3) ..
	waves.model.template shifted ( 0,3) ..
	waves.model.template shifted ( 2,3);

for i=1 upto 3:
	waves.model[i] = (waves.model.template shifted (0,-i*d));
endfor;

path mountain.l, mountain.r, mountain.m;
	% levi vrh Triglava
	mountain.l = ((0,3)--(-1.5,5)--(-3,3));
	% srednji vrh Triglava
	mountain.m = ((1.5,3)--(0,6)--(-1.5,3));
	% desni del Triglava
	mountain.r = mountain.l shifted (3,0);
	% sestavljeni Triglav
	waves.top =
		(mountain.r cutafter mountain.m)--
		(mountain.m cutbefore mountain.r cutafter mountain.l)--
		(mountain.l cutbefore mountain.m)--
		waves.model[0]--
		cycle;

	waves.mid = buildcycle (
		arc_lower.l[0],
		waves.model[2],
		arc_lower.r[0],
		reverse waves.model[1]);
	waves.bot = buildcycle ((reverse border[0]), (waves.model[3]));

%========%
% ZVEZDE %
%========%
% 1 2
%  0
path star[], star[].bigcirc, star[].smallcirc, star.model;
pair star[].c;
% premer zunanje kroznice okrog zvezde
numeric star.r; star.r = .45;

% sredisci zvezd 0 in 2 lezita na podaljsku levega pobocja Triglava
star[0].c = whatever[(-3,3),(-1.5,5)];
star[2].c = whatever[(-3,3),(-1.5,5)];
% sredisci zvezd 0 in 1 lezita na podaljsku desnega pobocja Triglava
star[0].c = whatever[( 3,3),( 1.5,5)];
star[1].c = whatever[( 3,3),( 1.5,5)];

% zvezdi 1 in 2 sta na isti visini
ypart star[1].c = ypart star[2].c;
xpart star[2].c + star.r = 1.5;

% model zvezde
star.model = (
	for i=0 upto 11:
		(i mod 2 + 1)*dir(i*30) --
	endfor cycle ) scaled .5star.r;

% model zvezde, premaknjen v pravo sredisce
for i=0 upto 2:
	star[i] = star.model shifted star[i].c;
	% zunanji in notranji krog, ki obkrozata zvezde
	star[i].bigcirc = fullcircle scaled 2star.r shifted star[i].c;
	star[i].smallcirc = fullcircle scaled star.r shifted star[i].c;
endfor;



%def draw_coa_bw(expr colblack, colwhite) =
%	fill border[2] scaled unit withcolor colwhite;
%	fill border[1] scaled unit withcolor colblack;
%	for i=0 upto 2:
%		fill star[i] scaled unit withcolor colwhite;
%	endfor;
%	fill waves.top scaled unit withcolor colwhite;
%	fill waves.mid scaled unit withcolor colwhite;
%	fill waves.bot scaled unit withcolor colwhite;
%enddef;

picture coa;


\stopMPinclusions

\starttext
\startMPpage
%----------------------%
% geometrijsko pravilo %
%----------------------%
%beginfig(1);
	drawoptions(withpen pencircle linewidth);
	draw_basic_grid;
	draw arc_upper scaled unit;
%	draw arc_lower.l[0] scaled unit;
	draw arc_lower.l[1] scaled unit;
	draw arc_lower.l[2] scaled unit;
%	draw arc_lower.r[0] scaled unit;
	draw arc_lower.r[1] scaled unit;
	draw arc_lower.r[2] scaled unit;

	draw mountain.l scaled unit;
	draw mountain.m scaled unit;
	draw mountain.r scaled unit;
%	draw mountain scaled unit;

	for i=0 upto 4:
		draw circle_waves[i] scaled unit;
	endfor;

	draw arc_upper.helpline scaled unit dashed evenly;
	% podaljski gora proti zvezdam
	draw ((3,3)--(-1.5,9)) scaled unit dashed evenly;
	draw ((-3,3)--(1.5,9)) scaled unit dashed evenly;
	% spodnji loki
	draw (((1.5,4)--(-5.5,2)) cutafter arc_lower.l[2]) scaled unit dashed evenly;
	draw border.l[2] scaled unit;
	draw border.l[1] scaled unit;
	draw border.r[2] scaled unit;
	draw border.r[1] scaled unit;
	
	% zvezde
	for i=0 upto 2:
		% veliki in mali rob
		draw star[i].bigcirc scaled unit;
		draw star[i].smallcirc scaled unit;
		% zvezda
		draw star[i] scaled unit;
	endfor;

	draw waves.bot scaled unit;
	draw waves.mid scaled unit;

	draw ((-3,3)--(3,3)) shifted (0,-1d) scaled unit;
	draw ((-3,3)--(3,3)) shifted (0,-2d) scaled unit;
	draw ((-3,3)--(3,3)) shifted (0,-3d) scaled unit;

%	draw (origin--dir(-120)) shifted (-2,3-d) scaled unit;

%endfig;
\stopMPpage

%------------%
% barvni grb %
%------------%
\startMPpage
%beginfig(2);
	fill border[2] scaled unit withcolor \MPcolor{si-red};
	fill border[1] scaled unit withcolor \MPcolor{si-blue};
	for i=0 upto 2:
		fill star[i] scaled unit withcolor \MPcolor{si-yellow};
	endfor;
	unfill waves.top scaled unit;
	unfill waves.mid scaled unit;
	unfill waves.bot scaled unit;

	coa := currentpicture;
%endfig;
\stopMPpage
\stoptext

